\documentclass[11pt]{article}

% ===================== PACKAGES =====================
\usepackage[a4paper,margin=1in]{geometry}
\usepackage{amsmath,amssymb}
\usepackage{enumitem}
\usepackage{fancyhdr}
\usepackage{xcolor}
\usepackage{titlesec}

% ===================== HEADER & FOOTER =====================
\pagestyle{fancy}
\fancyhf{}
\lhead{CSE 400: Fundamentals of Probability in Computing}
\rhead{Milestone 1 Scribe}
\cfoot{\thepage}

% ===================== SECTION FORMATTING =====================
\titleformat{\section}{\Large\bfseries}{}{0em}{}
\titleformat{\subsection}{\large\bfseries}{}{0em}{}
\titleformat{\subsubsection}{\normalsize\bfseries}{}{0em}{}

% ===================== TITLE =====================
\title{
    \normalsize School of Engineering and Applied Science (SEAS), Ahmedabad University \\
    \vspace{0.2cm}
    \textbf{CSE 400: Fundamentals of Probability in Computing}\\
    \Large Milestone 1 Scribe Submission
}
\author{}
\date{}

\begin{document}
\maketitle

\vspace{-2cm}
\begin{center}
    \begin{tabular}{ll}
        \textbf{Group:} & {s1\_g14\_net\hspace{3.5in}} \\ [1.5ex]
        \textbf{Project:} & {Handover Probability in Drone Cellular Networks \hspace{0.5in}} \\ [1.5ex]
        \textbf{Date of Submission:} & {\today \hspace{2.5in}}
    \end{tabular}
\end{center}

\hrule
\vspace{0.5cm}

% ========== SCRIBE QUESTION 1 ==========
\section{Scribe Question 1: Project System and Objective}

\subsection{What is the probabilistic problem being addressed in your project?}

The paper tackles handover issues in drone-based wireless networks. Here, drones are used as base stations rather than traditional cell towers. When a user's device switches connection from one drone to another, this is called a handover.

\vspace{0.3cm}

Traditional cellular networks have fixed base stations, but drones move constantly. Because of this moving, even stationary users on the ground will experience handovers as drones fly overhead and past them. The paper's goal is finding out how likely these handovers are within specific time periods.

\subsection{System Objective}

The main objective here is calculating handover probability - written as $P[H(t)]$. Basically, if someone connects to a drone at time zero $t = 0$, what are the chances they'll be connected to a different drone by time $t$? For example, after 10 seconds, will the user still have the same drone, or will they have switched? That's what the researchers want to figure out mathematically.

\subsection{Primary Sources of Uncertainty}

\begin{enumerate}[label=\arabic*.]
    \item \textbf{Where drones start}

    Nobody knows exactly where each drone begins. The researchers use something called a Poisson Point Process (PPP) with density $\lambda_0$ to model this. Think of it like randomly throwing rice on a plate - the grains land wherever they land. Same idea is followed with drone positions.

    \vspace{0.2cm}

    \item \textbf{Flight directions}

    Every drone flies straight, but which direction? That's completely random. North, south, northeast, whatever - any direction is equally possible. So there's no way to know which way a particular drone will go.

    \vspace{0.2cm}

    \item \textbf{Speed variations}

    The DSM (Different Speed Model) has drones flying at different speeds. Some drones will go faster, others slower. The paper uses probability distributions like Rayleigh or Uniform to handle this randomness.

    \vspace{0.2cm}

    \item \textbf{Which drone serves the user}

    Users connect to whichever drone is nearest. But drones are moving in random directions at potentially different speeds, so the nearest drone keeps changing. Predicting exactly when a new drone becomes closest is not possible.
\end{enumerate}

\newpage
