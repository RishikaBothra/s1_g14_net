\documentclass[11pt]{article}

% ===================== PACKAGES =====================
\usepackage[a4paper,margin=1in]{geometry}
\usepackage{amsmath,amssymb}
\usepackage{enumitem}
\usepackage{fancyhdr}
\usepackage{xcolor}
\usepackage{titlesec}

% ===================== HEADER & FOOTER =====================
\pagestyle{fancy}
\fancyhf{}
\lhead{CSE 400: Fundamentals of Probability in Computing}
\rhead{Milestone 1 Scribe}
\cfoot{\thepage}

% ===================== SECTION FORMATTING =====================
\titleformat{\section}{\Large\bfseries}{}{0em}{}
\titleformat{\subsection}{\large\bfseries}{}{0em}{}
\titleformat{\subsubsection}{\normalsize\bfseries}{}{0em}{}

% ===================== TITLE =====================
\title{
    \normalsize School of Engineering and Applied Science (SEAS), Ahmedabad University \\
    \vspace{0.2cm}
    \textbf{CSE 400: Fundamentals of Probability in Computing}\\
    \Large Milestone 1 Scribe Submission
}
\author{}
\date{}

\begin{document}
\maketitle

\vspace{-2cm}
\begin{center}
    \begin{tabular}{ll}
        \textbf{Group:} & {s1\_g14\_net\hspace{3.5in}} \\ [1.5ex]
        \textbf{Project:} & {Handover Probability in Drone Cellular Networks \hspace{0.5in}} \\ [1.5ex]
        \textbf{Date of Submission:} & {\today \hspace{2.5in}}
    \end{tabular}
\end{center}

\hrule
\vspace{0.5cm}

% ========== SCRIBE QUESTION 1 ==========
\section{Scribe Question 1: Project System and Objective}

\subsection{What is the probabilistic problem being addressed in your project?}

The paper tackles handover issues in drone-based wireless networks. Here, drones are used as base stations rather than traditional cell towers. When a user's device switches connection from one drone to another, this is called a handover.

\vspace{0.3cm}

Traditional cellular networks have fixed base stations, but drones move constantly. Because of this moving, even stationary users on the ground will experience handovers as drones fly overhead and past them. The paper's goal is finding out how likely these handovers are within specific time periods.

\subsection{System Objective}

The main objective here is calculating handover probability - written as $P[H(t)]$. Basically, if someone connects to a drone at time zero $t = 0$, what are the chances they'll be connected to a different drone by time $t$? For example, after 10 seconds, will the user still have the same drone, or will they have switched? That's what the researchers want to figure out mathematically.

\subsection{Primary Sources of Uncertainty}

\begin{enumerate}[label=\arabic*.]
    \item \textbf{Where drones start}

    Nobody knows exactly where each drone begins. The researchers use something called a Poisson Point Process (PPP) with density $\lambda_0$ to model this. Think of it like randomly throwing rice on a plate - the grains land wherever they land. Same idea is followed with drone positions.

    \vspace{0.2cm}

    \item \textbf{Flight directions}

    Every drone flies straight, but which direction? That's completely random. North, south, northeast, whatever - any direction is equally possible. So there's no way to know which way a particular drone will go.

    \vspace{0.2cm}

    \item \textbf{Speed variations}

    The DSM (Different Speed Model) has drones flying at different speeds. Some drones will go faster, others slower. The paper uses probability distributions like Rayleigh or Uniform to handle this randomness.

    \vspace{0.2cm}

    \item \textbf{Which drone serves the user}

    Users connect to whichever drone is nearest. But drones are moving in random directions at potentially different speeds, so the nearest drone keeps changing. Predicting exactly when a new drone becomes closest is not possible.
\end{enumerate}

\newpage

% ========== SCRIBE QUESTION 2 ==========
\section{Scribe Question 2: Key Random Variables and Uncertainty Modeling}

\subsection{Key Random Variables}

\subsubsection{1. Drone Positions - $\Phi_D(t)$}

This tracks where all the drones are at time $t$. Initially at $t=0$, they're scattered according to a Poisson Point Process with density $\lambda_0$. On average, there might be $\lambda_0$ drones per square kilometer, but we don't know the exact spots and that's the random part.

\vspace{0.3cm}

\subsubsection{2. Direction - $\theta$}

Each drone picks a direction $\theta$ to fly in. It's uniformly random anywhere from 0 to 360 degrees (or 0 to $2\pi$ in radians). The paper writes this as $\theta \sim U[0, 2\pi)$. So, here every direction has equal probability, making individual drone paths unpredictable.

\vspace{0.3cm}

\subsubsection{3. Speed - $V$}

There are two scenarios here:
\begin{itemize}
    \item \textbf{SSM (Same Speed Model):} In this every drone goes the same speed $v$, so no randomness
    \item \textbf{DSM (Different Speed Model):} In this speed $V$ varies randomly between drones using distributions like Rayleigh or Uniform
\end{itemize}

\vspace{0.3cm}

\subsubsection{4. Distance to serving drone - $u^*(t)$}

This tells how far the user is from serving drone at time $t$. As drones move around, this distance changes constantly. The uncertain part here is we can't predict which drone will be closest at any moment.

\vspace{0.3cm}

\subsubsection{5. User locations - $\Phi_U$}

Where users are positioned on the ground, also modeled with a Poisson Point Process. But here mostly focuses on one user at the origin, so matters less overall.

\vspace{0.4cm}

\subsection{How They Model Uncertainty}

The researchers use stochastic geometry as their framework.

\vspace{0.3cm}

\textbf{Poisson Point Process (PPP):}

Used for random drone positions. The idea is simple that we know roughly how many drones per area on average ($\lambda_0$), but exact locations are random. Key assumption: each drone's position doesn't influence where other drones are.

\vspace{0.3cm}

\textbf{Uniform distributions:}

For directions, since no direction should naturally be more likely than others when drones aren't coordinating.

\vspace{0.3cm}

\textbf{Speed distributions:}

In DSM, they picked Rayleigh or Uniform distributions for speeds. We're not entirely sure yet why these specific distributions were chosen over others.

\vspace{0.4cm}

\subsection{What They're Assuming}

\begin{enumerate}[label=\arabic*.]
    \item \textbf{Even distribution} - Drones spread out evenly on average

    \vspace{0.2cm}

    \item \textbf{No coordination} - Each drone does its own thing, there is no coordination between drones

    \vspace{0.2cm}

    \item \textbf{Straight paths} - They move just in straight lines no u-turns or curves

    \vspace{0.2cm}

    \item \textbf{Same height} - All drones fly at height $h$. None higher, none lower. This makes the 3D problem into a 2D one

    \vspace{0.2cm}

    \item \textbf{Nearest connection} - Users connect to the closest drone. Makes sense since closer usually means stronger signal

    \vspace{0.2cm}

    \item \textbf{Stationary users} - People stay put on the ground. This simplifies the math by removing user mobility from the equation
\end{enumerate}

\newpage

