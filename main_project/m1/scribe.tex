\documentclass[11pt]{article}

% ===================== PACKAGES =====================
\usepackage[a4paper,margin=1in]{geometry}
\usepackage{amsmath,amssymb}
\usepackage{enumitem}
\usepackage{fancyhdr}
\usepackage{xcolor}
\usepackage{titlesec}

% ===================== HEADER & FOOTER =====================
\pagestyle{fancy}
\fancyhf{}
\lhead{CSE 400: Fundamentals of Probability in Computing}
\rhead{Milestone 1 Scribe}
\cfoot{\thepage}

% ===================== SECTION FORMATTING =====================
\titleformat{\section}{\Large\bfseries}{}{0em}{}
\titleformat{\subsection}{\large\bfseries}{}{0em}{}
\titleformat{\subsubsection}{\normalsize\bfseries}{}{0em}{}

% ===================== TITLE =====================
\title{
    \normalsize School of Engineering and Applied Science (SEAS), Ahmedabad University \\
    \vspace{0.2cm}
    \textbf{CSE 400: Fundamentals of Probability in Computing} \\
    \Large Milestone 1 Scribe Submission
}
\author{}
\date{}

\begin{document}
\maketitle

\vspace{-2cm}
\begin{center}
    \begin{tabular}{ll}
        \textbf{Group:} & {s1\_g14\_net\hspace{3.5in}} \\ [1.5ex]
        \textbf{Project:} & {Handover Probability in Drone Cellular Networks \hspace{0.5in}} \\ [1.5ex]
        \textbf{Date of Submission:} & {\today \hspace{2.5in}}
    \end{tabular}
\end{center}

\hrule
\vspace{0.5cm}

% ========== SCRIBE QUESTION 1 ==========
\section{Scribe Question 1: Project System and Objective}

\subsection{What is the probabilistic problem being addressed in your project?}

So, the main problem which the paper is solving is about handovers in drone networks. Here the network in our phone is not coming from the towers but from the drones. Handovers happen when your phone switches from connecting to one drone to another drone.

\vspace{0.3cm}

The tricky part is that regular cell towers don't move, but drones do move around. So even if we are just standing still on the ground, our phone might need to switch between drones because they're flying around. The paper wants to find out what are the chances this switching or handover will happen in a certain amount of time.

\subsection{System Objective}

What they're actually trying to do is calculate handover probability. This is just the chance that our phone will switch to a different drone before some time $t$. Like let's suppose we start connected to one drone, what's the probability that 10 seconds later we will be connected to a different one? That's what the paper wants to figure out mathematically.

\subsection{Primary Sources of Uncertainty}

\begin{enumerate}[label=\arabic*.]
    \item \textbf{Where drones start} - Nobody knows the exact starting position of each drone. The paper uses something they call a Poisson Point Process (PPP) which just means the drones are placed randomly in the sky. Like if you threw a handful of drones up there without aiming, or like throwing a handful of rice on a plate which are further placed randomly.

    \vspace{0.2cm}

    \item \textbf{Which way they go} - Each drone flies in a straight line but the direction is totally random. Could be north, south, east, west, or anything in between. So it's again uncertain to know which direction a specific drone picked.

    \vspace{0.2cm}

    \item \textbf{How fast they go} - In one of the models called DSM, different drones can fly at different speeds. Some might be going faster, some slower. They use probability distributions to handle this randomness.

    \vspace{0.2cm}

    \item \textbf{Which drone you connect to} - Your phone always connects to whichever drone is closest. But since drones are moving and also in random directions, the closest one keeps changing and we can't predict exactly when a different drone will become the closest.
\end{enumerate}

\newpage
\documentclass[11pt]{article}

%

% ========== SCRIBE QUESTION 2 ==========
\section{Scribe Question 2: Key Random Variables and Uncertainty Modeling}

\subsection{Key Random Variables}

\subsubsection{Where the drones are - $\Phi_D(t)$}

\begin{itemize}
    \item This is basically tracking where all the drones are at any time $t$
    \item When things start at $t=0$, the drones are scattered around following this Poisson Point Process (PPP) thing with density $\lambda_0$
    \item It's like if you randomly dropped drones on a map, you'd expect about $\lambda_0$ drones in each square kilometer on average
    \item We don't know the exact spots, that's the uncertain part
\end{itemize}

\vspace{0.3cm}

\subsubsection{Which direction they fly - $\theta$}

\begin{itemize}
    \item Each drone picks a direction $\theta$ to fly
    \item It's completely random, any angle from 0 to 360 degrees
    \item The paper states it as $\theta \sim U[0, 2\pi)$ which just means uniform random
    \item So this makes it totally unpredictable
\end{itemize}

\vspace{0.3cm}

\subsubsection{How fast drones fly - $V$}

\begin{itemize}
    \item In SSM (Same Speed Model): all drones go the same speed $v$, so no randomness here
    \item In DSM (Different Speed Model): $V$ is random and changes for different drones
    \item They tried Rayleigh distribution or uniform distribution for this
    \item This captures how some drones might be faster or slower than others
\end{itemize}

\vspace{0.3cm}

\subsubsection{Distance to your drone - $u^*(t)$}

\begin{itemize}
    \item This is how far you are from the drone serving you at time $t$
    \item It keeps changing as drones fly around
    \item The uncertain part is we don't know which drone will be closest at any moment
\end{itemize}

\vspace{0.3cm}

\subsubsection{Where people are - $\Phi_U$}

\begin{itemize}
    \item This is where phones and devices on the ground are located
    \item Also uses Poisson Point Process
    \item But they mostly focus on one person at the center, so this matters less
\end{itemize}

\vspace{0.4cm}

\subsection{How They Model Uncertainty}

The paper uses something called stochastic geometry. We are still trying to understand what this fully means but:

\vspace{0.2cm}

\begin{itemize}
    \item \textbf{Poisson Point Process:} They use this for random drone positions. It's basically saying ``we expect this many drones per area on average, but we don't know exactly where.'' One big assumption is that each drone's position doesn't affect others.

    \vspace{0.2cm}

    \item \textbf{Uniform distributions:} They use this for directions since no direction should be more likely than any other.

    \vspace{0.2cm}

    \item \textbf{Speed distributions:} In DSM, they pick either Rayleigh or uniform distributions to show that drones have different speeds. Not sure yet why they picked these specific distributions.
\end{itemize}

\vspace{0.4cm}

\subsection{What They're Assuming}

\begin{enumerate}[label=\arabic*.]
    \item \textbf{Drones spread evenly} - On average the drones are spread out nicely, not all bunched up in one area

    \vspace{0.2cm}

    \item \textbf{Drones don't coordinate} - Each drone moves on its own without caring what other drones are doing

    \vspace{0.2cm}

    \item \textbf{Straight paths} - Drones fly straight and don't turn during the time we're watching

    \vspace{0.2cm}

    \item \textbf{Same altitude} - All drones fly at height $h$, none higher or lower

    \vspace{0.2cm}

    \item \textbf{Connect to nearest} - Your phone always connects to the closest drone, which makes sense

    \vspace{0.2cm}

    \item \textbf{People don't move} - The people on the ground stay put, which makes the math easier
\end{enumerate}

\newpage
