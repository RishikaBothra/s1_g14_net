\documentclass[11pt]{article}

% ===================== PACKAGES =====================
\usepackage[a4paper,margin=1in]{geometry}
\usepackage{amsmath,amssymb}
\usepackage{enumitem}
\usepackage{fancyhdr}
\usepackage{xcolor}
\usepackage{titlesec}

% ===================== HEADER & FOOTER =====================
\pagestyle{fancy}
\fancyhf{}
\lhead{CSE 400: Fundamentals of Probability in Computing}
\rhead{Milestone 1 Scribe}
\cfoot{\thepage}

% ===================== SECTION FORMATTING =====================
\titleformat{\section}{\Large\bfseries}{}{0em}{}
\titleformat{\subsection}{\large\bfseries}{}{0em}{}
\titleformat{\subsubsection}{\normalsize\bfseries}{}{0em}{}

% ===================== TITLE =====================
\title{
    \normalsize School of Engineering and Applied Science (SEAS), Ahmedabad University \\
    \vspace{0.2cm}
    \textbf{CSE 400: Fundamentals of Probability in Computing}\\
    \Large Milestone 1 Scribe Submission
}
\author{}
\date{}

\begin{document}
\maketitle

\vspace{-2cm}
\begin{center}
    \begin{tabular}{ll}
        \textbf{Group:} & {s1\_g14\_net\hspace{3.5in}} \\ [1.5ex]
        \textbf{Project:} & {Handover Probability in Drone Cellular Networks \hspace{0.5in}} \\ [1.5ex]
        \textbf{Date of Submission:} & {\today \hspace{2.5in}}
    \end{tabular}
\end{center}

\hrule
\vspace{0.5cm}

% ========== SCRIBE QUESTION 1 ==========
\section{Scribe Question 1: Project System and Objective}

\subsection{What is the probabilistic problem being addressed in your project?}

The paper tackles handover issues in drone-based wireless networks. Here, drones are used as base stations rather than traditional cell towers. When a user's device switches connection from one drone to another, this is called a handover.

\vspace{0.3cm}

Traditional cellular networks have fixed base stations, but drones move constantly. Because of this moving, even stationary users on the ground will experience handovers as drones fly overhead and past them. The paper's goal is finding out how likely these handovers are within specific time periods.

\subsection{System Objective}

The main objective here is calculating handover probability - written as $P[H(t)]$. Basically, if someone connects to a drone at time zero $t = 0$, what are the chances they'll be connected to a different drone by time $t$? For example, after 10 seconds, will the user still have the same drone, or will they have switched? That's what the researchers want to figure out mathematically.

\subsection{Primary Sources of Uncertainty}

\begin{enumerate}[label=\arabic*.]
    \item \textbf{Where drones start}

    Nobody knows exactly where each drone begins. The researchers use something called a Poisson Point Process (PPP) with density $\lambda_0$ to model this. Think of it like randomly throwing rice on a plate - the grains land wherever they land. Same idea is followed with drone positions.

    \vspace{0.2cm}

    \item \textbf{Flight directions}

    Every drone flies straight, but which direction? That's completely random. North, south, northeast, whatever - any direction is equally possible. So there's no way to know which way a particular drone will go.

    \vspace{0.2cm}

    \item \textbf{Speed variations}

    The DSM (Different Speed Model) has drones flying at different speeds. Some drones will go faster, others slower. The paper uses probability distributions like Rayleigh or Uniform to handle this randomness.

    \vspace{0.2cm}

    \item \textbf{Which drone serves the user}

    Users connect to whichever drone is nearest. But drones are moving in random directions at potentially different speeds, so the nearest drone keeps changing. Predicting exactly when a new drone becomes closest is not possible.
\end{enumerate}

\newpage

% ========== SCRIBE QUESTION 2 ==========
\section{Scribe Question 2: Key Random Variables and Uncertainty Modeling}

\subsection{Key Random Variables}

\subsubsection{1. Drone Positions - $\Phi_D(t)$}

This tracks where all the drones are at time $t$. Initially at $t=0$, they're scattered according to a Poisson Point Process with density $\lambda_0$. On average, there might be $\lambda_0$ drones per square kilometer, but we don't know the exact spots and that's the random part.

\vspace{0.3cm}

\subsubsection{2. Direction - $\theta$}

Each drone picks a direction $\theta$ to fly in. It's uniformly random anywhere from 0 to 360 degrees (or 0 to $2\pi$ in radians). The paper writes this as $\theta \sim U[0, 2\pi)$. So, here every direction has equal probability, making individual drone paths unpredictable.

\vspace{0.3cm}

\subsubsection{3. Speed - $V$}

There are two scenarios here:
\begin{itemize}
    \item \textbf{SSM (Same Speed Model):} In this every drone goes the same speed $v$, so no randomness
    \item \textbf{DSM (Different Speed Model):} In this speed $V$ varies randomly between drones using distributions like Rayleigh or Uniform
\end{itemize}

\vspace{0.3cm}

\subsubsection{4. Distance to serving drone - $u^*(t)$}

This tells how far the user is from serving drone at time $t$. As drones move around, this distance changes constantly. The uncertain part here is we can't predict which drone will be closest at any moment.

\vspace{0.3cm}

\subsubsection{5. User locations - $\Phi_U$}

Where users are positioned on the ground, also modeled with a Poisson Point Process. But here mostly focuses on one user at the origin, so matters less overall.

\vspace{0.4cm}

\subsection{How They Model Uncertainty}

The researchers use stochastic geometry as their framework.

\vspace{0.3cm}

\textbf{Poisson Point Process (PPP):}

Used for random drone positions. The idea is simple that we know roughly how many drones per area on average ($\lambda_0$), but exact locations are random. Key assumption: each drone's position doesn't influence where other drones are.

\vspace{0.3cm}

\textbf{Uniform distributions:}

For directions, since no direction should naturally be more likely than others when drones aren't coordinating.

\vspace{0.3cm}

\textbf{Speed distributions:}

In DSM, they picked Rayleigh or Uniform distributions for speeds. We're not entirely sure yet why these specific distributions were chosen over others.

\vspace{0.4cm}

\subsection{What They're Assuming}

\begin{enumerate}[label=\arabic*.]
    \item \textbf{Even distribution} - Drones spread out evenly on average

    \vspace{0.2cm}

    \item \textbf{No coordination} - Each drone does its own thing, there is no coordination between drones

    \vspace{0.2cm}

    \item \textbf{Straight paths} - They move just in straight lines no u-turns or curves

    \vspace{0.2cm}

    \item \textbf{Same height} - All drones fly at height $h$. None higher, none lower. This makes the 3D problem into a 2D one

    \vspace{0.2cm}

    \item \textbf{Nearest connection} - Users connect to the closest drone. Makes sense since closer usually means stronger signal

    \vspace{0.2cm}

    \item \textbf{Stationary users} - People stay put on the ground. This simplifies the math by removing user mobility from the equation
\end{enumerate}

\newpage

% ========== SCRIBE QUESTION 2 ==========
\section{Scribe Question 2: Key Random Variables and Uncertainty Modeling}

\subsection{Key Random Variables}

\subsubsection{1. Drone Positions - $\Phi_D(t)$}

This tracks where all the drones are at time $t$. Initially at $t=0$, they're scattered according to a Poisson Point Process with density $\lambda_0$. On average, there might be $\lambda_0$ drones per square kilometer, but we don't know the exact spots and that's the random part.

\vspace{0.3cm}

\subsubsection{2. Direction - $\theta$}

Each drone picks a direction $\theta$ to fly in. It's uniformly random anywhere from 0 to 360 degrees (or 0 to $2\pi$ in radians). The paper writes this as $\theta \sim U[0, 2\pi)$. So, here every direction has equal probability, making individual drone paths unpredictable.

\vspace{0.3cm}

\subsubsection{3. Speed - $V$}

There are two scenarios here:
\begin{itemize}
    \item \textbf{SSM (Same Speed Model):} In this every drone goes the same speed $v$, so no randomness
    \item \textbf{DSM (Different Speed Model):} In this speed $V$ varies randomly between drones using distributions like Rayleigh or Uniform
\end{itemize}

\vspace{0.3cm}

\subsubsection{4. Distance to serving drone - $u^*(t)$}

This tells how far the user is from serving drone at time $t$. As drones move around, this distance changes constantly. The uncertain part here is we can't predict which drone will be closest at any moment.

\vspace{0.3cm}

\subsubsection{5. User locations - $\Phi_U$}

Where users are positioned on the ground, also modeled with a Poisson Point Process. But here mostly focuses on one user at the origin, so matters less overall.

\vspace{0.4cm}

\subsection{How They Model Uncertainty}

The researchers use stochastic geometry as their framework.

\vspace{0.3cm}

\textbf{Poisson Point Process (PPP):}

Used for random drone positions. The idea is simple that we know roughly how many drones per area on average ($\lambda_0$), but exact locations are random. Key assumption: each drone's position doesn't influence where other drones are.

\vspace{0.3cm}

\textbf{Uniform distributions:}

For directions, since no direction should naturally be more likely than others when drones aren't coordinating.

\vspace{0.3cm}

\textbf{Speed distributions:}

In DSM, they picked Rayleigh or Uniform distributions for speeds. We're not entirely sure yet why these specific distributions were chosen over others.

\vspace{0.4cm}

\subsection{What They're Assuming}

\begin{enumerate}[label=\arabic*.]
    \item \textbf{Even distribution} - Drones spread out evenly on average

    \vspace{0.2cm}

    \item \textbf{No coordination} - Each drone does its own thing, there is no coordination between drones

    \vspace{0.2cm}

    \item \textbf{Straight paths} - They move just in straight lines no u-turns or curves

    \vspace{0.2cm}

    \item \textbf{Same height} - All drones fly at height $h$. None higher, none lower. This makes the 3D problem into a 2D one

    \vspace{0.2cm}

    \item \textbf{Nearest connection} - Users connect to the closest drone. Makes sense since closer usually means stronger signal

    \vspace{0.2cm}

    \item \textbf{Stationary users} - People stay put on the ground. This simplifies the math by removing user mobility from the equation
\end{enumerate}

\newpage

% ========== SCRIBE QUESTION 3 ==========
\section{Scribe Question 3: Probabilistic Reasoning and Dependencies}

\subsection{Independence and Dependence}

The paper assumes drones are independent from each other spatially. Why does this matter?

\vspace{0.3cm}

Because it lets them use the displacement theorem. This theorem says that if drones start at random spots and then move independently, they'll still be randomly positioned after moving. That makes the math way easier to work with.

\vspace{0.3cm}

But there's dependence over time for a single drone. Where a drone is now affects where it'll be later. If it's flying in a straight line, knowing its current position tells us something about future positions.

\vspace{0.4cm}

\subsection{Conditioning}

Conditioning shows up a lot in the paper - calculating probabilities when certain things are known. We've noticed two main uses:

\vspace{0.3cm}

\textbf{First:} In Theorem 2, they condition on initial distance $r$ and direction $\theta$. They figure out handover probability for specific values of these, then average over all possible values.

\vspace{0.3cm}

\textbf{Second:} There's an ``exclusion zone'' concept. If a user is connected to one drone (the serving drone), no other drone can be inside a certain area around the user. Why? Because if another drone were closer, the user would be connected to that one instead. So only the serving drone can be in this zone.

\vspace{0.4cm}

\subsection{The Equivalence (Theorem 1)}

This is pretty interesting. The paper shows two scenarios are mathematically equivalent:
\begin{itemize}
    \item Mobile drones with stationary users (what they're studying)
    \item Static base stations with mobile users (regular cellular networks)
\end{itemize}

\vspace{0.3cm}

This works because of something called translation invariance in Poisson Point Processes. From what we understand, if all points in a PPP shift by the same amount, the statistical properties don't change - it still looks random. We're working on understanding this concept better.

\vspace{0.4cm}

\subsection{Outcome}

\subsubsection{Current Understanding}

The paper has methods for calculating handover probability if the drone's starting position and elapsed time are known.

Two inputs:
\begin{itemize}
    \item Where the drone started
    \item How much time passed
\end{itemize}

\vspace{0.3cm}

\subsubsection{Predicting Changes Over Time}

Figure 2 shows how handover probability changes as time passes. The pattern is clear: more time means higher handover probability. This makes sense because more time means more drone movement, which means more chances for handovers.

\vspace{0.3cm}

\subsubsection{Comparing SSM vs DSM}

DSM shows lower handover probability than SSM.

\vspace{0.3cm}

\subsubsection{Lower Bounds}

For DSM, the paper only gives a lower bound on handover probability, not an exact answer. We need to understand what ``lower bound'' means here and why they can't get exact values like they do for SSM.

\vspace{0.4cm}

\subsection{Main Dependencies}

Three types of dependency show up in the analysis:

\begin{enumerate}[label=\arabic*.]
    \item \textbf{Spatial dependencies} - Which drone serves a user depends on which drone is nearest

    \vspace{0.2cm}

    \item \textbf{Temporal dependencies} - A drone's position at any time depends on where it started and how it's been moving

    \vspace{0.2cm}

    \item \textbf{Independence} - Different drones move independently for making the problem solvable
\end{enumerate}

\newpage

% ========== SCRIBE QUESTION 4 ==========
\section{Scribe Question 4: Model--Implementation Alignment}

\subsection{How the Model Aligns with Implementation}

The researchers didn't build actual drone networks. Instead, they ran Monte Carlo simulations, looking at average behavior across many random scenarios with different drone positions, directions, and speeds.

\vspace{0.3cm}

The mathematical model and simulation line up like this:

\vspace{0.3cm}

\textbf{Mathematical side:}

PPP defines how drones are distributed. All movements are random. Performance is measured by averaging across random deployments of drones.

\vspace{0.3cm}

\textbf{Simulation side:}

Drones get placed randomly based on PPP. Each time step updates positions using assigned speeds and directions. Simulation runs until all handover events are recorded.

\vspace{0.4cm}

\subsection{Key Assumptions}

\subsubsection{Nearest-neighbor rule}

Users always connect to the closest drone. While in real systems consider signal quality, interference, load balancing, but this assumption simplifies things for both math and simulation.

\vspace{0.3cm}

\subsubsection{Straight-line movement}

Real drones turn, hover, avoid obstacles. The model assumes purely linear paths, which makes calculations easy.

\vspace{0.3cm}

\subsubsection{Single altitude}

All drones at height $h$. This converts a 3D problem into 2D, dramatically simplifying calculations.

\vspace{0.3cm}

\subsubsection{Independent movement}

No coordination between drones. Each can be simulated separately, which supports the independence assumption mathematically.

\newpage

% ========== SCRIBE QUESTION 5 ==========
\section{Scribe Question 5: Cross-Milestone Consistency and Change}

\subsection{Current Status}

We're early in understanding this paper. Some parts are clearer than others. A lot is still tentative and needs more investigation.

\vspace{0.4cm}

\subsection{What's reasonably clear}

\textbf{The problem:}

Finding handover probability in drone-based networks.

\vspace{0.3cm}

\textbf{Two models:}
\begin{itemize}
    \item SSM (Same Speed Model)
    \item DSM (Different Speed Model)
\end{itemize}

\vspace{0.3cm}

\textbf{Random variables involved:}
\begin{itemize}
    \item Drone locations
    \item Movement directions
    \item Speeds
\end{itemize}

\vspace{0.3cm}

\textbf{Association rule:}

Users connect to nearest drone.

\vspace{0.4cm}

\subsection{What Needs More Work}

Several concepts need deeper understanding:

\begin{itemize}
    \item Why use PPP specifically for drone locations?
    \item How does the displacement theorem actually work here?
    \item What does translation invariance really mean mathematically?
    \item How does Theorem 1's equivalence actually hold?
    \item How were Theorem 2's exact values derived?
    \item Why does DSM only give bounds instead of exact answers?
    \item What role of Theorem 3?
\end{itemize}

\vspace{0.4cm}

\subsection{What We Expect to Learn}

\begin{itemize}
    \item The mathematical proofs will become clearer
    \item We'll understand simulation implementation better
    \item Model limitations will become more apparent
\end{itemize}

\newpage

% ========== SCRIBE QUESTION 6 ==========
\section{Scribe Question 6: Open Issues and Responsibility Attribution}

\subsection{Unresolved Questions}

\subsubsection{About PPP}

\begin{itemize}
    \item Why is this the right model for drone positions?
    \item How would 3D movement change things?
\end{itemize}

\vspace{0.3cm}

\subsubsection{About Theorem 1}

\begin{itemize}
    \item What makes mobile drones with static users equivalent to static towers with mobile users?
\end{itemize}

\vspace{0.3cm}

\subsubsection{About DSM}

\begin{itemize}
    \item Why can't we get exact formulas like we can for SSM?
    \item What does ``lower bound'' actually mean for handover probability?
    \item How close is this bound to real values?
\end{itemize}

\vspace{0.3cm}

\subsubsection{Real-world concerns}

\begin{itemize}
    \item Could actual drone networks use this model?
    \item What physical constraints matter like battery life, obstacles, no-fly zones?
    \item How do we choose between SSM and DSM in practice?
\end{itemize}

\vspace{0.4cm}

\subsection{Next Steps}

\begin{itemize}
    \item Reread the paper, focusing on mathematical sections
    \item Learn about Poisson Point Processes
    \item Find additional resources for difficult concepts
\end{itemize}

\vspace{0.5cm}

\bigskip
\hrule
\vspace{0.2cm}
\begin{center}
    \textit{End of Milestone 1 Scribe}
\end{center}

\end{document}