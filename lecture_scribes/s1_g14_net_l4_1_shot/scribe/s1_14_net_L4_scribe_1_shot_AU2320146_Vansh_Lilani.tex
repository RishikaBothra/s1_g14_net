\documentclass[11pt]{article}
\usepackage[utf8]{inputenc}
\usepackage{amsmath, amssymb, amsthm}
\usepackage{geometry}
\geometry{a4paper, margin=1in}

\title{Lecture Scribe: Joint Probability and Conditional Probability \\
\large Course: CSE 400 — Fundamentals of Probability in Computing}
\author{Vansh Lilani (ID: AU2320146)}
\date{January 15, 2026}

\begin{document}

\maketitle

\section{Topic Title}
Joint Probability and Conditional Probability

\section{Definitions and Notation}
\begin{itemize}
    \item \textbf{Experiment ($E$):} A procedure performed that produces a specific result. \\
    \textit{Example:} Tossing a coin five times ($E_{5}$).
    
    \item \textbf{Outcome ($\xi$):} A possible result of an experiment. \\
    \textit{Example:} One possible outcome of $E_{5}$ is $\xi_{1} = HHTHT$.
    
    \item \textbf{Sample Space ($S$):} The set of all possible outcomes of an experiment.
    
    \item \textbf{Event:} A subset of the sample space.
    
    \item \textbf{Joint Probability:} The probability of two or more events occurring simultaneously.
    
    \item \textbf{Conditional Probability:} The probability of an event occurring given that another event has already occurred.
\end{itemize}

\section{Assumptions / Conditions}
\textbf{Axioms of Probability:} All probability assignments must satisfy the fundamental axioms:
\begin{enumerate}
    \item \textbf{Non-negativity:} $P(A) \geq 0$ for any event $A$.
    \item \textbf{Normalization:} $P(S) = 1$.
    \item \textbf{Additivity:} For mutually exclusive events, the probability of their union is the sum of their individual probabilities.
\end{enumerate}

\noindent \textbf{Probability Assignment Approaches:}
\begin{itemize}
    \item \textbf{Classical Approach:} Assumes all outcomes in a finite sample space are equally likely.
    \item \textbf{Relative Frequency Approach:} Based on the limit of the frequency of an outcome over many trials.
\end{itemize}

\section{Main Results / Theorems}
\begin{itemize}
    \item \textbf{Joint Probability Notation:} Represented as $P(A \cap B)$ or $P(A, B)$, indicating the probability that both Event A and Event B occur.
    
    \item \textbf{Conditional Probability Formula:} The probability of event $A$ given event $B$ is defined as:
    \[ P(A|B) = \frac{P(A \cap B)}{P(B)}, \text{ where } P(B) > 0 \]
\end{itemize}

\section{Proofs / Derivations}
\textit{Note: Focus is on conceptual motivation in engineering contexts (e.g., Speech Recognition, Radar Systems).}

\subsection*{Derivation of Multiplication Rule}
\begin{enumerate}
    \item \textbf{Step 1:} Start with the definition of conditional probability: 
    \[ P(A|B) = \frac{P(A \cap B)}{P(B)} \]
    
    \item \textbf{Step 2:} Rearrange the formula to solve for the joint probability by multiplying both sides by $P(B)$.
    
    \item \textbf{Step 3:} The resulting \textbf{Multiplication Rule} is: 
    \[ P(A \cap B) = P(A|B)P(B) \]
\end{enumerate}

\section{Worked Examples}
\begin{itemize}
    \item \textbf{Example 1: Card Deck (Joint Probability):} Calculation of probabilities involving specific suits and ranks from a standard deck.
    \item \textbf{Example 2: Costume Party (Joint Probability):} Determining the likelihood of overlapping characteristics among party guests.
    \item \textbf{Example 3: Cards Without Replacement (Conditional Probability):} Calculating the probability of drawing a specific sequence of cards when the first card is not returned to the deck.
    \item \textbf{Example 4: Game of Poker:} Applying conditional probability to determine the strength of a hand as more cards are revealed.
    \item \textbf{Example 5: The Missing Key:} A logical probability problem determining the likelihood of finding a key in a specific location given it was not found in others.
\end{itemize}

\vfill
\begin{flushleft}
    \textbf{Reference:} Lecture 4 - Joint Probability and Conditional Probability by Dhaval Patel, PhD.
\end{flushleft}

\end{document}